\documentclass[12pt,a4paper]{article}
% \usepackage[english]{babel}
% \usepackage[utf8x]{inputenc}

\usepackage{graphicx} % Required for inserting images.
\usepackage[margin=25mm]{geometry}
\parskip 4.2pt  % Sets spacing between paragraphs.
\renewcommand{\baselinestretch}{1.5} 
\parindent 8.4pt  % Sets leading space for paragraphs.
\usepackage[font=sf]{caption} % Changes font of captions.

\usepackage{amsmath}
\usepackage{amsfonts}
\usepackage{amssymb}
\usepackage{siunitx}
\usepackage{verbatim}
\usepackage{hyperref} % Required for inserting clickable links.
\usepackage{natbib} % Required for APA-style citations.
\usepackage{sectsty}
\usepackage[document]{ragged2e}
\subsectionfont{\centering}

\title{Programming Proofs Project}
\author{Fady Adal}

\begin{document}
\maketitle

I thoroughly enjoying working on the project, though I do not think I was successful with it. What started as a hope to formalize the compilation steps of the compiler outlind by \cite{morrisett1999} ended up as an incomplete formalization of System F using intrinsic types and De Brujin indices. With lots of \texttt{admit}s in stuck proofs, this project explored how using De Brujin indices, while theoritically convenient, can become harder to think about in more information-rich scenarios, and the numerous weakening and substitution lemmas become more and more specific. Moreover, it explored how having an intrinically-typed term sounds nice on paper with the inability to construct a type-incorrect term (and thus releaving the need for a preservation proof!) this upfront cost can be hindering. If I were to restart, I would have probably also ended up using intrinsic types and De Brujin indices (especially the former), though.

This project is based heavily on \cite{chapman2019}, but with syntax made to resemble the System F outlined in the Typed Assembly paper. While the former paper gives a full outline of how one formalizes an intrinsically typed System F, they do not show a lot of important lemmas that I had to proof on my own. Moreover, while I found that they're very close, I had to translate different semantics from Agda (the paper) to lean (the project).

Overall, I enjoyed interacting with lean working on this project. I don't think I will abandon this project, since I'm still intrigued about the idea of having a verified compiler to typed assembly (using intrinsic types along the different intermediate calculi), and so will probably work on it during near future.

\bibliographystyle{apalike}
\bibliography{commentary}



\end{document}
